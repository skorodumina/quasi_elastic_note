\newpage
\chapter{Conclusion}
\mbox{}\vspace{-\baselineskip}

A sophisticated testing of several parameterizations~\cite{Bosted_fit,Bosted:2007xd,Durand:1961zz,Kocevar:1967} existing for the deuteron quasi-elastic peak was performed via the comparison of the parameterized cross sections with two published sets of experimental cross sections~\cite{Hanson:1973vf,Rock:1991jy,Rock_SLAC}. The comparison was made in the wide $Q^{2}$ range from $\sim$0.3~GeV$^{2}$ to $\sim$4~GeV$^{2}$. This impartial examination allows to make the following conclusions.


\begin{itemize}

\item The Bosted parameterization~\cite{Bosted_fit,Bosted:2007xd} in its default implementation systematically overestimates the measured integral cross sections under the quasi-elastic peak. The overall data description quality gradually decreases from several percent to almost 20\% as $Q^2$ grows from 0.3~GeV$^{2}$ to 4~GeV$^{2}$.

\item The Bosted parameterization~\cite{Bosted_fit,Bosted:2007xd} with $F_{QE}$ calculated according to Eq.~\eqref{eq:fqe_scaling} systematically underestimates the measured integral cross sections under the quasi-elastic peak. The overall data description quality gradually increases from $\sim$15\% to a few percent as $Q^2$ grows from 0.3~GeV$^{2}$ to 4~GeV$^{2}$.

\item The normalization of the parametrized distributions to the values provided by the corresponding peak cross section approximations~\cite{Durand:1961zz,Kocevar:1967} gives some improvement in the description quality for the majority of considered measurements.


\end{itemize}


 
Thus, in the considered $Q^{2}$ range the Bosted parameterization of the deuteron quasi-elastic peak (in its default implementation) was found to give worse data description quality than that offered by the Bosted parameterization of the proton elastic peak (see App.~\ref{app_epelas}). This may serve as an indication that internal structure of the deuteron is not yet fully understood. 



In addition to this examination, some helpful tools that may be of use for studying quasi-elastic and inclusive cross sections off deuterons are given in App.~\ref{app_tools}.



