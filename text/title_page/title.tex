\noindent\begin{minipage}{\textwidth}
\begin{center}
\thispagestyle{empty}
\vspace{0.5cm}
{ \Large{Testing Parameterizations of the Deuteron Quasi-Elastic Peak}}\\
\vspace{1cm}

{\large Iu.A. Skorodumina$^{1, a}$, G.V. Fedotov$^{2}$,  R.W. Gothe$^{1}$} \\[16pt]

\parbox{.86\textwidth}{\centering\footnotesize\it
$^1$Department of Physics and Astronomy, University of South Carolina, Columbia, SC\\[8pt]
\setstretch{0.3} 
$^2$National Research Centre ``Kurchatov Institute" B. P. Konstantinov Petersburg Nuclear Physics Institute, Gatchina, St. Petersburg, Russia\\
[20pt]
E-mail: $^a$skorodum@jlab.org}\\


\vspace{2cm}
{\bf Abstract}\\[9pt]

\end{center}
{\small This study introduces common parameterizations of the quasi-elastic peak in the electron scattering spectrum off deuterium and provides the comparison of the parameterized cross sections with published experimental data. The comparison is performed in the wide $Q^{2}$ range from $\sim$0.3~GeV$^{2}$ to $\sim$4~GeV$^{2}$. In this way the performance of the parameterizations and their ability to describe experimental measurements are impartially tested and the conclusion on the description reliability is made.}


\end{minipage}

