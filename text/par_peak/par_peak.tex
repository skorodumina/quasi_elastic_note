\newpage
\chapter{Approximations of the peak cross section value}
\mbox{}\vspace{-\baselineskip}

The first consistent description of inelastic electron-deuteron scattering, which in contrast to earlier calculations took into account all important contributions to the electron-nucleon interaction as well as to the final state interactions between the outgoing nucleons, was Durand's theory~\cite{Durand:1959zz,Durand:1961zz}. In 1961 in Ref.~\cite{Durand:1961zz} Durand gave a simple approximation formula for the cross section at the quasi-elastic peak, which then was being widely used by experimentalists. This formula is given by
\begin{equation}
\left [ \frac{d^{2}\sigma}{d\Omega dE'} \right ]_{peak} = \left [ \frac{d\sigma}{d\Omega} \right ]_{Mott}^{*} ( 4.57\cdot 10^{-3}) \frac{m_{N}^{2}}{pE} \left (G_{E_{p}}^{2} + G_{E_{n}}^{2} + \frac{\tau}{\epsilon}\left [G_{M_{p}}^{2} + G_{M_{n}}^{2}\right ]  \right )\frac{1}{1+\tau},\label{eq:durand}
\end{equation}
where $\epsilon = \left (1 +2(1+\tau)\tan^{2}\frac{\theta_{e'}}{2} \right )^{-1}$, $p = q/2$ with the photon laboratory momentum magnitude $q$, $E=\sqrt{p^{2} + m_{N}^{2}}$, and all other quantities are defined as above\footnote[3]{The quantity $p$ here corresponds to the momentum of the final proton in the cms of outgoing nucleons. See discussions in Refs.~\cite{Durand:1959zz,Durand:1961zz,Budnitz:1969dt} on the validity of the approximation $p = q/2$ at the quasi-elastic peak.}. The numerical coefficient is in MeV$^{-1}$.

%, which is supposed to give a reasonable result for photon virtualities $<2$~GeV$^{2}$,

Since Durand's theory a lot of papers on inelastic electron-deuteron scattering~\cite{Kocevar:1967,Budnitz:1969dt,Hanson:1973vf} have tried to modify it with respect to one or the other point to get a still better understanding of the existing experimental data. Among them a very interesting is the study~\cite{Kocevar:1967}, which estimates some higher order contributions that stem from the use of complete relativistic kinematics and provides a different formula for the peak cross section (see Eq.~(50) in Ref.~\cite{Kocevar:1967})\footnote[4]{When using the peak cross section formulae in this study, the nucleon electromagnetic form factors $G_{E}$ and $G_{M}$ are estimated according to Ref.~\cite{Bosted:1994tm}.}.



