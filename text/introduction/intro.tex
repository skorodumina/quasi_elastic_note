

\newpage
\chapter{Introduction}
\mbox{}\vspace{-\baselineskip}



The elastic cross section is one of the most universal and well-understood characteristics of a scattering process. This observable, depending only on a few parameters, is easily measurable experimentally and can be relatively easy parameterized, and therefore represents a rather popular quantity of scientific interest. Being an important observable itself, the elastic cross section also often serves as a reference point for various complicated analyses of exclusive reactions with multiparticle final states, where this observable is extracted as an auxiliary quantity in order to verify both the correct normalization of the main result and the quality of the electron selection.

Nowadays the elastic cross section of electron scattering off a free proton is well-known\footnote[1]{See App.~\ref{app_formalism} for some details on the $ep$ scattering formalism.} over a wide kinematic range, since it has been intensively studied experimentally for decades and eventually almost perfectly described by parameterizations as that of Peter Bosted. This parameterization is based on the study from Ref.~\cite{Bosted:1994tm}, which provides an empirical fit to the world data for the proton elastic electromagnetic form factors in the range 0~GeV$^2<Q^2<$ 30~GeV$^2$ and to the neutron electromagnetic form factors in the range 0~GeV$^2<Q^2<$ 10~GeV$^2$. The ability of the Bosted parameterization to describe available experimental data on the elastic $ep$ cross section is demonstrated in Tab.~\ref{tab:epelast} of App.~\ref{app_epelas}. This table provides the comparison between measured cross section values taken from Refs.~\cite{Goitein:1970pz,Sill:1992qw,Christy:2004rc} and the corresponding parameterized values. The comparison reveals an excellent agreement between the experimental measurements and the parameterization within a few percent, which has a tendency to slightly worsen as $Q^{2}$ grows from $\sim$0.3~GeV$^{2}$ to $\sim$5~GeV$^{2}$.

It is also noteworthy that in several analyses of exclusive reactions off the free proton the comparison of the auxiliary extracted elastic cross sections with the Bosted parameterization was performed in order to check the overall normalization of analyzed observables as well as the quality of the electron selection~\cite{Fed_an_note:2007,Fedotov:2008aa,Fed_an_note:2017,Fedotov:2018oan,Isupov_note,Isupov:2017lnd}. Here, the study~\cite{Fed_an_note:2007,Fedotov:2008aa} performed for 0.2~GeV$^2<Q^{2}<$ 0.6~GeV$^2$ observed agreement between experimental and parameterized values within better than 5\%, the study~\cite{Fed_an_note:2017, Fedotov:2018oan} performed for 0.4~GeV$^2<Q^{2}<$ 1.0~GeV$^2$ observed $\sim$3\% agreement, while the study~\cite{Isupov_note,Isupov:2017lnd} performed for 2~GeV$^2<Q^{2}<$ 5~GeV$^2$ observed $\sim$10\% agreement.


Meanwhile, for electron scattering conducted off a nucleus the corresponding quantity of interest is the quasi-elastic cross section off nucleons. In contrast with the elastic spectrum off the free proton, which is discrete for a given beam energy and at fixed polar scattering angle, the quasi-elastic cross section off nucleons is continuously spread over the energy of the scattered electron. This smearing, caused by the motion of nucleons within a nucleus, forms a so-called quasi-elastic peak in the scattering spectrum. The position and shape of this peak contain information about the internal structure of nuclei.

Compared to elastic scattering off free protons, quasi-elastic scattering off nucleons in nuclei is less understood and lacking the same quality of theoretical description. Nonetheless, several techniques have been developed on this matter with the deuteron (as the lightest nucleus) being the most investigated target. 

This note introduces several existing parameterizations for the deuteron quasi-elastic peak and provides the comparison of the parameterized cross sections with published experimental data. The comparison is performed in the wide $Q^{2}$ range from $\sim$0.3~GeV$^{2}$ to $\sim$4~GeV$^{2}$. In this way the performance of the studied parameterizations and their ability to describe experimental measurements are impartially tested and the conclusion on the description reliability is made.


This examination, already interesting by itself, can be of great use for those deuteron target analyses that use the auxiliary extracted quasi-elastic cross section as a reference point in order to verify both the correct normalization of the main result and the quality of the electron selection.



